\documentclass[11pt]{article}

\usepackage{fullpage,times,namedplus,pgf, amsmath}
\DeclareMathOperator*{\argmax}{arg\,max}

\newcommand{\myroot}{\ensuremath{\mathrm{root}}}

\title{Simulate binary-state epigenome evoluiton}

\begin{document}
\maketitle

Assume the epigenome is a sequence of auto-correlated sequence of
binary-state random variables. The auto-correlation reflects the
organization of the epigenome as alternating domains bearing a certain
modification or not. Let the epigenomic sequence be $S=s_1s_2\ldots
s_N$. As a graphical model, neighboring sites are connected with
undirected edges. $S$ evolves over time. Assume the stationary
distribution for the epigenome has a Gibbs distribution that factorize
over pairs of neighboring sites:
\[
\Pr(S) = \frac{1}{Z} \exp\big\{\phi(s_1) +\sum_{n=1}^{N-1}\phi(s_n, s_{n+1}) + \phi(s_N)\big\}
\]

The evolution of an individual site is context-dependent. The
instantaneous mutation rate from state $s$ to the alternative
state $\bar{s}$ is $\gamma(l, s, r)$, where $l$, $s$ and $r$
are the states of three consecutive sites.  For a time interval $[0,t)$,
given that the states of $l$ and $r$ are not changed, then
the states of site $s$ follows a continouse-time Markov chain, 
and the holding time thus follows an exponential distribution.
An observation of the path can be summarized with
\[
  L = \big\{s(0), k, \{t_i\}_{i=1}^{k}, t \big\},
\]
where $s(0)$ is the state at time 0, $k$ is the total number of jumps,
$t_i$ is the time when the $i$-th jump occurred, and $t$ is the total
length of the time interval.

Suppose $L_l$ and $L_r$ are given paths of two neighboring sites, then
the union of their jumping times
\[
  \{0, t\} \cup \{t_{li}\}_{i=1}^{k_l} \cup \{t_{ri}\}_{i=1}^{k_r} 
\] defines time intervals, within each of which
the states of $l$ and $r$ stayed constant.

\paragraph{Simulation scheme} We are going to simulate
a full history of epigenome evolution for a time interval $[0, t]$.

\begin{enumerate}
\item Simulate the starting methylome using a binary-state Markov model
\item Initialize all paths $L_n = \{s_n(0), k_n=0, T_n=\emptyset, t\}$, for $n=1,\ldots, N.$
\item (For simplicity, fix the paths $L_1$ and $L_N$ as initialized.) For
  site $n = 2, \ldots, N-1$, simulate $L_n$ given the current paths of
  $L_{n-1}$ and $L_{n+1}$:
  \begin{itemize}
  \item Collect the time intervals from site $n-1$ and site $n+1$, so
    that within each of the intervals the states of the neighboring
    sites are unchanged. Let the intervals be represented by a sorted
    array $\{t_0, t_1, \cdots, t_M\}$.
  \item  Let $x$ be a random variable from an exponential
    distribution, representing the jumping time of the middle site given
    that the states of its two neighbors are constant.  For a time
    interval $[t_{m-1}, t_m]$ during which the neighboring sites' states
    are unchanged, suppose $x\sim \text{Exp}(\lambda)$. The parameter
    $\lambda$ is a function of current state and the states of the two
    neighboring sites.
  \item Generate a binary observation $I$ with probability $p = \Pr(x < t_m- t_{m-1})$
  \item If $I=1$, a jump happens within the time interval $[t_{m-1},
    t_m]$.
    \begin{enumerate}
      \item[(1)] Add 1 to the number of jumps $k_n \leftarrow k_n +1$.
      \item[(2)] Generate a random variable from the truncated exponential
        distribution, for example using importance sampling. Let the sample
        value be $t' \in (0, t_m -t_{m-1})$. Update $T_n \leftarrow
        T_n\cup\{t_{m-1}+t'\}$.
  \item[(3)] Then update $t_{m-1}$ with $t_{m-1}+t'$, i.e. the new
    time interval is $[t_{m-1}+t', t_m]$. Update the exponential parameter
    accordingly, because the current state has changed. Then repeat from
    the previous step in the outer loop.
    \end{enumerate}
  \item If $I=0$, there is no jump during this time interval, the
    middle site keeps its state unchanged till time $t_m$. If $t_m =
t_M$, STOP.  Otherwise, we move onto the next time interval $[t_m,
t_{m+1}]$ and repeat from the previous step in the outer loop.
  \end{itemize}
\item Repeat last step, until the epigenome summary statistics converge to
  a stable distribution.
\end{enumerate}

\paragraph{Relationship between mutation rates and stationary distribution}
What transition rate function $\gamma$ can lead to a stationary
distribution determined by $\phi$?  The Proposition 1 of
\citetext{jensen2000probabilistic} gives a sufficient condition:
\begin{equation}\label{eqn:prop1}
  \frac{\gamma(l, s, r)} {\gamma(l, \bar{s}, r)} = \frac{\exp(\phi(l, \bar{s})+ \phi(\bar{s}, r))}{\exp(\phi(l, s)+ \phi(s, r))},
\end{equation}
which is derived from the reversibility property of the stationary
distribution.

The proposition 2 and 3 give a way of specifying $\gamma$ from $\phi$.
Assume that the log intensities can be written as
\[
  \log(\gamma(l, s, r)) = -g(l, s, r) + \ell(l, r),
\]
and that there exists a function $q(l, r)$ such that
\[
g(l, s, r) = g(l, s, *) - g(l, *, *) + g(s, r, *) -g(s, *, *) + q(l, r)
\]
Then $g$ bridges $\gamma$ and $\phi$ with
\[
\phi(l, s) = g(l, s, *) - g(l, *, *),
\]
where `*' stands for averaged function value over all values of the
indicated operands. So we only need to specify function $g$, which has
8 possible input configurations. Based on empirical understanding of
epigenomes, we want $g$ (and $\gamma$) to have left-right symmetry,
i.e. $g(a, b, c) = g(c, b, a)$. Under this assumption, two pairs of
configurations are equivalent, leaving 6 distinct configurations. Let
the values of $g$ be as specified in table~\ref{tab:level}.
\begin{table}[t!]
  \centering
  \begin{tabular}{|p{4cm}|p{4cm}|p{4cm}|}\hline
     & \multicolumn{2}{|c|}{\textbf{Mutation type in patterns ($g$ parameter )}} \\\hline
    \textbf{$\gamma$ level} & 0$\rightarrow$ 1& 1$\rightarrow$ 0 \\\hline
    low & 0,0,0 ($x_1$)& 1,1,1 ($y_1$) \\\hline
    medium & 0,0,1 ($x_2$) & 1,1,0 ($y_2$) \\\hline
    medium & 1,0,0 ($x_2$)& 0,1,1 ($y_2$)\\\hline
    high & 1,0,1 ($x_3$)& 0,1,0 ($y_3$) \\\hline    
  \end{tabular}
  \caption{Level of mutation rates in different patterns}
  \label{tab:level}
\end{table}

However, we can directly verify that if we define mutation rates as follows,
\begin{equation}
\log (\gamma(l, s, r)) =  \ell(l, r) + (\phi(l, \bar{s}) +\phi(\bar{s}, r)),
\end{equation}
where $\ell$ is some function independent of $s$, then the rates
satisfy the condition in Equation~\ref{eqn:prop1}. 



\bibliographystyle{namedplus}
\bibliography{biblio}

\end{document}

 
